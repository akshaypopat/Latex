%!tex root: ../main.tex
\documentclass[../main.tex]{subfiles}

\begin{document}
  \begin{exercise}
    \begin{enumerate}[label=(\alph*)]
      \item Let $Y$ be the plane curve $y = x^2$ (i.e., $Y$ is the zero set of the polynomial $f = y - x^2$). Show that $A(Y)$ is isomorphic to a polynomial ring in one variable over $k$.
      \item Let $Z$ be the plane curve $xy = 1$. Show that $A(Z)$ is not isomorphic to a polynomial ring in one variable over $k$.
    \end{enumerate}
  \end{exercise}
  \begin{proof}
    \begin{enumerate}[label=(\alph*)]
      \item $y - x^2$ is irreducible, so $I(Y) = (y - x^2)$. Therefore $A(Y) = k[x,y]/(y-x^2) = k[x]$.
      \item Similarly, $A(Z) = k[x,y]/(xy - 1) = k[x, x^{-1}]$. Suppose there existed an isomorphism $\phi: k[x, x^{-1}] \to k[t]$. $x$, $x^{-1}$ and all non-zero elements of $k$ are units in $k[x,x^{-1}]$. Therefore, their images under $\phi$ must be units. However, $k[t]^\times = k \setminus \{0\}$, so there is no element of $k[x, x^{-1}]$ which maps to $t$, since the elements of $k[x, x^{-1}]$ are polynomials in $x$ and $x^{-1}$ and $\phi$ is a ring homomorphism. This contradicts the injectivity of $\phi$.
    \end{enumerate}
  \end{proof}

  \begin{exercise}
    \emph{The twisted cubic curve.} Let $Y \subseteq \mathbb{A}^3$ be the set $\{ (t, t^2, t^3) : t \in k \}$. Show that $Y$ is an affine variety of dimension $1$. Find generators for the ideal $I(Y)$. Show that $A(Y)$ is isomorphic to a polynomial ring in one variable over $k$. We say that $Y$ is given by the \emph{parametric representation} $x = t$, $y = t^2$, $z = t^3$.
  \end{exercise}
  \begin{proof}
    It is easy to verify that $Y = Z(y - x^2, z - x^3)$. Then $A/(y - x^2, z - x^3) = k[x]$, which is a principal ideal domain, so in particular an integral domain, meaning $(y - x^2, z - x^3)$ is prime, so $Y$ is an affine variety. Also, this tells us that $I(Y) = (y - x^2, z - x^2)$ since all prime ideals are radical, so $y - x^2$ and $z - x^2$ are generators for $I(Y)$. We already saw that $A(Y) = A/(y - x^2, z - x^3) = k[x]$. To see that $Y$ has dimension $1$, observe that $\dim A(Y) = \operatorname{trdeg}_k k(x) = 1$.
  \end{proof}

  \begin{exercise}
    Let $Y$ be the algebraic set in $\mathbb{A}^3$ defined by the two polynomials $x^2 - yz$ and $xz - x$. Show that  $Y$ is a union of three irreducible components. Describe them and find their prime ideals.
  \end{exercise}
  \begin{proof}
    \begin{align*}
      Y &= Z(x^2 - yz, xz - x) \\
        &= Z(x^2 - yz) \cap (Z(x) \cup Z(z - 1)) \\
        &= (Z(x^2 - yz) \cap Z(x)) \cup (Z(x^2 - yz) \cap Z(z - 1)) \\
        &= ((Z(y) \cup Z(z)) \cap Z(x)) \cup (Z(x^2 - y) \cap Z(z - 1)) \\
        &= (Z(y) \cap Z(x)) \cup (Z(z) \cap Z(x)) \cup (Z(x^2 - y) \cap Z(z - 1)) \\
        &= Z(x,y) \cup Z(x,z) \cup Z(x^2 - y, z - 1). \\
    \end{align*}
    These are irreducible because $(x,y)$, $(x,z)$ and $(x^2 - y, z - 1)$ are prime ideals of $k[x,y,z]$.
  \end{proof}

  \begin{exercise}
    If we identify $\mathbb{A}^2$ with $\mathbb{A}^1 \times \mathbb{A}^1$ in the natural way, show that the Zariski topology on $\mathbb{A}^2$ is not the product topology of the Zariski topologies on the two copies of $\mathbb{A}^1$.
  \end{exercise}
  \begin{proof}
    The complement of $Z(y - x)$ is open in $\mathbb{A}^2$ by definition, but can be seen not to be open in $\mathbb{A}^1 \times \mathbb{A}^1$. The topology on $\mathbb{A}^1 \times \mathbb{A}^1$ is generated by sets of the form $U \times V$, where $U$ and $V$ are open subsets of $\mathbb{A}^1$. An open subset of $\mathbb{A}^1$ is either empty or the complement of a finite set of points. However, $Z(y - x)$ is infinite, so its complement cannot be written as a union of sets of the form $U \times V$.
  \end{proof}

  \begin{exercise}
    Show that a $k$-algebra $B$ is isomorphic to the affine coordinate ring of some algebraic set in $\mathbb{A}^n$, for some $n$, if and only if $B$ is a finitely generated $k$-algebra with no nilpotent elements.
  \end{exercise}
  \begin{proof}
    We already know that for all algebraic sets $Y \subseteq \mathbb{A}^n$, the coordinate ring $A(Y)$ is finitely generated. We will show that $A(Y)$ has no nilpotent elements. Suppose $f \in A(Y)$ is nilpotent. We can think of $f$ as a polynomial such that $f^r \in I(Y)$ for some positive integer $r$. Since $Y$ is algebraic, $I(Y)$ is radical, so $f \in I(Y)$, meaning $f$ is the zero element of $A(Y)$.

    Now we show that if $B$ is a finitely generated $k$-algebra with no nilpotent elements then it is isomorphic to a coordinate ring. $B$ is finitely generated, say by $n$ elements, so there is a surjective homomorphism $\phi: A \to B$. It is surjective, so by the isomorphism theorem $A/\ker\phi \cong B$. Let $Y = Z(\ker\phi)$. $\phi$ is a homomorphism, so if $f^r \in \ker\phi$, then that means $\phi(f^r) = \phi(f)^r = 0$, implying $f \in \ker\phi$, since $B$ has no nilpotent elements. In other words $\ker\phi$ is a radical ideal. Therefore $A(Y) = A/I(Y) = A/\ker\phi \cong B$.
  \end{proof}

  \begin{exercise}
    Any non-empty open subset of an irreducible topological space is dense and irreducible. If $Y$ is subset of a topological space $X$, which is irreducible in its induced topology, then the closure $\overline{Y}$ is also irreducible.
  \end{exercise}
  \begin{proof}
    Let $Z$ be an irreducible topological space and $U \subseteq Z$ be non-empty and open. Let $V = Z \setminus U$ be the complement of $U$. $V$ is closed. We can write $Z = \overline{U} \cup V$, a union of closed sets. $Z$ is irreducible, so either $\overline{U} = Z$ or $V = Z$. $U$ was non-empty, so we cannot have $V = Z$. Therefore $\overline{U} = Z$. Suppose, for contradiction, that $U$ is not irreducible. Then we could write $U = U_1 \cup U_2$, where $U_1$ and $U_2$ are closed proper subsets of $U$. Then $U_1$ and $U_2$ are also closed in $Z$, and therefore their union $U$ is closed in $Z$. Then we could write $Z = U \cup V$. Since $Z$ is irreducible, either $U = Z$ or $V = Z$, but $U$ is non-empty so we know $V \ne Z$. Therefore $U = Z$, but then $U$ is irreducible.
    
    Suppose, for contradiction, that $\overline{Y}$ is not irreducible. Then we can write $\overline{Y} = Y_1 \cup Y_2$, where $Y_1$ and $Y_2$ are closed proper subsets of $\overline{Y}$. For $i = 1,2$, $X \setminus \overline{Y}$ and $\overline{Y} \setminus Y_i$ are open, so $X \setminus Y_i = (X \setminus Y) \cup (\overline{Y} \setminus Y_i)$ is open. Therefore $Y_i$ is closed, so $Y \cap Y_i$ is closed in $Y$. Then, we can write $Y = (Y \cap Y_1) \cup (Y \cap Y_2)$. Irreducibility of $Y$ implies, without loss of generality, that $Y \subseteq Y_1$. Overall we have $Y \subseteq Y_1 \subsetneq \overline{Y}$, a contradiction, since $\overline{Y}$ is the minimal closed set containing $Y$.
  \end{proof}
\end{document}
