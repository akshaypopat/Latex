%!tex root: ./main.tex
\documentclass[./main.tex]{subfiles}

\begin{document}
We fix an algebraically closed field $k$ which we work over for the entirety of this chapter, and let $A = k[x_1, \dots, x_n]$.

  \section{Algebraic sets and radical ideals}
  
  \begin{definition}
    The \emph{zero set} for an ideal $\mathfrak{a} \subseteq A$ is the set
    \[
      Z(\mathfrak{a}) = \left\{ P \in k^n \; \middle| \; f(P) = 0 \text{ for all } f \in \mathfrak{a} \right\}.
    \]
  \end{definition}

  Any collection of polynomials generates an ideal, and any ideal in $A$ is finitely generated. The zero set of a collection of polynomials is then the zero set of the ideal they generate, which is exactly the zeros of those polynomials. Also, the zero set of an ideal is exactly the zeros any generating set of that ideal. We denote, for example, the zero set of a polynomial $f$ by $Z(f)$, and the zero set of a collection of polynomials $T$ by $Z(T)$.

  \begin{definition}
    A subset $Y \subseteq k^n$ is \emph{algebraic} if it is a zero set.
  \end{definition}

  \begin{definition}
    \emph{Affine $n$-space} over $k$, denoted $\mathbb{A}^n$, is the space $k^n$ equipped with the \emph{Zariski topology}, the topology where open sets are complements of algebraic sets. Equivalently, it is the topology where the closed sets are exactly the algebraic sets.
  \end{definition}

  \begin{example}
    $\mathbb{A}^2 \ncong \mathbb{A}^1 \times \mathbb{A}^1$.
  \end{example}

  \begin{definition}
    The \emph{ideal} of a subset $Y \subseteq \mathbb{A}^n$, denoted $I(Y)$, is the set of polynomials that vanish on $Y$. That is,
    \[
      I(Y) = \left\{ f \in A \; \middle| \; f(P) = 0 \text{ for all } P \in Y \right\}.
    \]
  \end{definition}

  Taking zeros and ideals are not exactly inverses, but they are almost inverses. We make the relationship more precise.

  \begin{theorem}[Hilbert's Nullstellensatz]
    For any ideal $\mathfrak{a} \subseteq A$, if $f \in I(Z(\mathfrak{a}))$ then $f^r \in \mathfrak{a}$ for some non-negative integer $r$.
  \end{theorem}

  In other words, $I(Z(\mathfrak{a}))$ is the \emph{radical} of $\mathfrak{a}$:

  \begin{definition}
    In a ring $R$, the \emph{radical} of an ideal $\mathfrak{a}$ is
    \[
      \sqrt \mathfrak{a} = \left\{ f \in R \; \middle| \; f^r \in \mathfrak{a}\text{ for some non-negative integer } r\right\}.
    \]
    We say that an ideal is \emph{radical} if it is equal to its radical.
  \end{definition}

  Furthermore, we have the following properties.

  \begin{proposition}
    \begin{enumerate}[label=(\roman*)]
      \item For any subsets $S \subseteq T \subseteq A$ we have $Z(T) \subseteq Z(S)$
      \item For any subsets $X \subseteq Y \subseteq \mathbb{A}^n$ we have $I(Y) \subseteq I(X)$
      \item For any collection of subsets $T_\alpha \subseteq A$ we have $Z(\bigcup_\alpha T_\alpha) = \bigcap_\alpha Z(T_\alpha)$
      \item For any subsets $T_1, T_2 \subseteq A$ we have $Z(T_1) \cup Z(T_2) = Z(T_1 T_2)$
      \item For any collection of subsets $Y_\alpha \subseteq \mathbb{A}^n$ we have $I(\bigcup_\alpha Y_\alpha) = \bigcap_\alpha I(Y_\alpha)$
      \item For any subset $Y \subseteq \mathbb{A}^n$ we have $Z(I(Y)) = \overline{Y}$
    \end{enumerate}
  \end{proposition} 
  \begin{proof}
    (i)-(v) are obviously fine. We prove (vi). $Z(I(Y))$ is closed and obviously contains $Y$, so it is clear that $\overline{Y} \subseteq Z(I(Y))$. All that remains is to show the reverse inclusion $Z(I(Y)) \subseteq \overline{Y}$.

    $\overline{Y}$ is closed, so we can write $\overline{Y} = Z(\mathfrak{a})$ for some ideal $\mathfrak{a} \subseteq A$. Then we know $Y \subseteq Z(\mathfrak{a})$ so $I(Y) \supseteq I(Z(\mathfrak{a})) \supseteq \mathfrak{a}$. Applying $Z$ to $\mathfrak{a} \subseteq I(Y)$ gives the result.
  \end{proof}

  So although we do not quite have a bijective correspondence between subsets of $\mathbb{A}^n$ and ideals of $A$, we do have a bijective correspondence of \emph{closed} subsets and \emph{radical} ideals, given by $Y \mapsto I(Y)$ and $\mathfrak{a} \mapsto Z(\mathfrak{a})$.

  \section{Irreducibility}
  
  \begin{definition}
    A non-empty subset $Y$ of a topological space $X$ is \emph{irreducible} if it cannot be written as a union $Y = Y_1 \cup Y_2$, where $Y_1$ and $Y_2$ are both proper closed subsets of $Y$.
  \end{definition}

  \begin{proposition}
    Let $X$ be a topological space. $X$ is irreducible if and only if all non-empty open subsets of $X$ are dense.
  \end{proposition}
  \begin{proof}
    Let $U$ be a non-empty open subset of $X$. Let $V = X \setminus U$. Write $X = \overline{U} \cup V$, a union of closed sets. $X$ is irreducible so $X = \overline{U}$ or $X = V$. $U$ is non-empty, so $X \ne V$. Therefore $\overline{U} = X$.

    Let $X$ be a topological space such that all non-empty open subsets are dense. Suppose, for contradiction, that $X$ is not irreducible. Then we can write $X = X_1 \cup X_2$, where $X_1$ and $X_2$ are closed proper subsets. Then $(X \setminus X_1) \cap (X \setminus X_2) = \varnothing$, so $(X \setminus X_1) \subseteq X_2$. Since $(X \setminus X_1)$ is open and non-empty, it is dense, so $X \subseteq X_2$, a contradiction.
  \end{proof}

  \begin{definition}
    A subset $Y \subseteq \mathbb{A}^n$ is an \emph{affine variety} if it is closed and irreducible.
  \end{definition}

  \begin{proposition}
    In a commutative ring $R$, all prime ideals are radical.
  \end{proposition}
  \begin{proof}
    If $\mathfrak{p}$ is a prime ideal and $f \in \sqrt \mathfrak{p}$, then either $f \in \mathfrak{p}$ or $f^{r-1} \in \mathfrak{p}$. Repeat this for $f^{r-1}$ and so on, and we see that $f \in \mathfrak{p}$.
  \end{proof}

  \begin{proposition}
    A closed subset of $Y \subseteq \mathbb{A}^n$ is irreducible if and only if its corresponding ideal $I(Y)$ is prime. Equivalently, we could say that a radical ideal $\mathfrak{a}$ is prime if and only if its zero set $Z(\mathfrak{a})$ is irreducible.
  \end{proposition}
  \begin{proof}
    Assume $Y$ is irreducible. If $fg \in I(Y)$ then $Z(fg) \supseteq Y$ ($Y$ is closed), so $Y \subseteq Z(f) \cup Z(g)$. Therefore we can write $Y$ as a union of closed sets $Y = (Z(f) \cap Y) \cup (Z(g) \cap Y)$. $Y$ is irreducible so $Y = Z(f) \cap Y$ without loss of generality. This means $Y \subseteq Z(f)$. Therefore $f \in I(Y)$.

    Conversely, assume that $\mathfrak{a}$ is a prime ideal (and therefore also radical). If $Z(\mathfrak{a}) = Z(\mathfrak{a}_1) \cup Z(\mathfrak{a}_2)$, then $Z(\mathfrak{a}) = Z(\mathfrak{a}_1 \mathfrak{a}_2)$. This means $\mathfrak{a} = \sqrt{\mathfrak{a}_1 \mathfrak{a_2}}$, so $\mathfrak{a}_1 \mathfrak{a}_2 \subseteq \mathfrak{a}$. Since $\mathfrak{a}$ is prime, we have either $\mathfrak{a}_1 \subseteq \mathfrak{a}$ or $\mathfrak{a}_2 \subseteq \mathfrak{a}$. Therefore $Z(\mathfrak{a})$ is irreducible.
  \end{proof}
  
  With this we can see our first examples of affine varieties. Notice how we use algebra argue about the geometric property of irreducibility.

  \begin{example}
    The zero ideal $(0) \subseteq A$ is prime, so $Z(0) = \mathbb{A}^n$ is irreducible.
  \end{example}

  \begin{example}
    If $f \in A$ is irreducible then since $A$ is a unique factorisation domain, $f$ is prime. Therefore $Z(f)$ is irreducible. 
  \end{example}

  \begin{nonexample}
    Let $k = \mathbb{R}$, a field which is not algebraically closed. Let $A = k[x]$. Then $I(Z(x^2 + 1)) = A \ne (x^2 + 1)$. The whole correspondence breaks down.
  \end{nonexample}

  \begin{definition}
    Let $Y$ be the zero set of a non-constant irreducible polynomial $f \in A$. If $n = 3$ then $Y$ is called a \emph{surface}. If $n > 3$ then $Y$ is called a \emph{hypersurface}.
  \end{definition}

  \begin{definition}
    A topological space $X$ is \emph{Noetherian} if all descending chains of closed subsets stabilise. That is, if
    \[
      Y_1 \supseteq Y_2 \supseteq \cdots
    \]
    are closed subsets of $X$, then there exists a positive integer $r$ such that $Y_r = Y_{r+1} = \cdots$.
  \end{definition}

  Descending, because the correspondence between subsets and ideals is inclusion reversing. The following example elaborates.

  \begin{example}
    $\mathbb{A}^n$ is Noetherian.
  \end{example}
  To see this, notice that any descending chain of closed subsets of $\mathbb{A}^n$ gives
  \[
    I(Y_1) \subseteq I(Y_2) \subseteq \cdots,
  \]
  which stabilises since $A$ is a Noetherian ring. Therefore by applying $Z$ we get that the original chain stabilises.

  \begin{proposition}
    Let $X$ be a Noetherian topological space. All closed subsets $Y \subseteq X$ have a unique (up to reordering and removing nested sets) decomposition into finitely many irreducible closed subsets
    \[
      Y = Y_1 \cup \cdots \cup Y_r, \quad Y_i \not\subseteq Y_j \text{ for } i \ne j.
    \]
  \end{proposition}
  \begin{proof}
    First we do existence, then we do uniqueness. $Y$ is either irreducible or not. If it is irreducible, we are done. If not, then write $Y = Y_1 \cup Y_2$. Each of $Y_1$ and $Y_2$ are either irreducible or not. If a subset is not irreducible, reduce it. Continue like this until there are only irreducible subsets left. Since $X$ is Noetherian, this process must terminate in finitely many steps.

    Suppose there is another decomposition $Y = Y'_1 \cup \cdots \cup Y'_s$. Then for all $i$, $Y_i$ is covered by these sets, so $Y_i = (Y_i \cap Y'_1) \cup \cdots \cup (Y_i \cap Y'_s)$. However, $Y_i$ is irreducible so $Y_i \subseteq Y'_j$ for some $j$. Similarly, $Y'_j \subseteq Y_k$ for some $k$. This means $Y_i \subseteq Y_k$, so $k=i$. Overall we have $Y_i \subseteq Y'_j \subseteq Y_i$, so $Y_i = Y'_j$. Therefore the two decompositions are the same up to reordering.
  \end{proof}

  So in particular, any closed subset of $\mathbb{A}^n$ can be written uniquely as a union of affine varieties.

  \section{Dimension}
  
  \begin{definition}
    For a closed subset $Y \subseteq \mathbb{A}^n$, the \emph{affine coordinate ring} of $Y$ is
    \[
      A(Y) = A/I(Y).
    \]
  \end{definition}

  If $Y$ is an affine variety then $A(Y)$ is an integral domain, and $A(Y)$ is a finitely-generated $k$-algebra. Conversely, any integral domain that is a finitely-generated $k$-algebra is the affine coordinate ring of some affine variety.

  The coordinate ring of a closed subset $Y$ can be thought of as the polynomials on $Y$. For example, if $Y = Z(f)$ for an irreducible polynomial $f \in A$, then $g \in A(Y)$ is a polynomial such that the variables satisfy the relation given by $f$. They are `in the world of $Y$'.

  \begin{definition}
    Let $X$ be a topological space. The \emph{dimension} of $X$, denoted $\dim X$, is the maximum non-negative integer $n$ such that there exists an ascending chain of distinct irreducible closed subsets of $X$
    \[
      Z_0 \subsetneq \cdots \subsetneq Z_n.
    \]
    The \emph{dimension} of an affine variety is its dimension as a topological space.
  \end{definition}
  We start from $Z_0$ because we want to allow spaces to be $0$-dimensional.

  \begin{definition}
    Let $A$ be a ring, and $\mathfrak{p} \subseteq A$ a prime ideal. The height of $\mathfrak{p}$, denoted $\operatorname{height} \mathfrak{p}$, is the maximum non-negative integer $n$ such that there exists an ascending chain of distinct prime ideals
    \[
      \mathfrak{p}_0 \subsetneq \cdots \subsetneq \mathfrak{p}_n = \mathfrak{p}.
    \]
  \end{definition}

  \begin{definition}
    Let $A$ be a ring. The \emph{(Krull) dimension} of $A$, denoted $\dim A$, is the maximum of the heights of all prime ideals of $A$. That is, it is the maximum non-negative integer $n$ such that there exists an ascending chain of distinct prime ideals
    \[
      \mathfrak{p}_0 \subsetneq \cdots \subsetneq \mathfrak{p}_n.
    \]
  \end{definition}

  \begin{proposition}
    Let $Y$ be a closed subset of $\mathbb{A}^n$. Its dimension is equal to the dimension of its coordinate ring. That is,
    \[
      \dim Y = \dim A(Y).
    \]
  \end{proposition}
  \begin{proof}
    There is a bijective correspondence between irreducible closed subsets of $Y$ and prime ideals containing $I(Y)$. There is also a bijective correspondence between prime ideals containing $I(Y)$ and prime ideals of $A(Y)$. Therefore the maximum length of a chain of prime ideals in $A(Y)$ is also the maximum length of a chain of closed irreducible subsets of $Y$.
  \end{proof}

  \begin{definition}
    Let $k \subseteq K$ be a field extension. Let $E = \{a_1, \dots, a_n\}$ be a non-empty finite subset of $K$. $E$ is \emph{algebraically independent} over $k$ if for all $f \in k[x_1, \dots, x_n]$, we have $f(a_1, \dots, a_n) \ne 0$. 
  \end{definition}

  \begin{definition}
    Let $k \subseteq K$ be a field extension. The \emph{transcendence degree} of $K$ over $k$, denoted $\operatorname{trdeg}_k K$, is the maximum possible cardinality of an algebraically independent subset of $K$.
  \end{definition}

  \begin{proposition}
    The transcendence degree of $\operatorname{frac}A$ over $k$ is $\operatorname{trdeg}_k k(x_1, \dots, x_n) = n$.
  \end{proposition}

  \begin{theorem}
    Let $k$ be a field and $B$ an integral domain which is a finitely-generated $k$-algebra. Then
    \begin{enumerate}[label=(\roman*)]
      \item $\dim B = \operatorname{trdeg}_k(\operatorname{frac}B)$
      \item For all prime ideals $\mathfrak{p} \subseteq B$, we have
        \[
          \operatorname{height}\mathfrak{p} + \dim(B/\mathfrak{p}) = \dim B
        \]
    \end{enumerate}
  \end{theorem}

  \begin{corollary}
    $\dim \mathbb{A}^n = n$.
  \end{corollary}
  \begin{proof}
    We know $\dim \mathbb{A}^n = \dim A(\mathbb{A}^n)$. Since $A(\mathbb{A}^n) = A$, which is an integral domain and a finitely generated $k$-algebra, we can apply (i) of the theorem along with the proposition to deduce the result.
  \end{proof}

  \begin{theorem}[Krull's Hauptidealsatz]
    Let \(A\) be a Noetherian ring and let \(f \in A\) be an element which is neither a zero-divisor nor a unit. Let \(\mathfrak{p}\) be a minimal prime ideal containing \(f\). Then \(\operatorname{height}\mathfrak{p} = 1\).
  \end{theorem}

  \begin{proposition}
    A Noetherian integral domain is a unique factorisation domain if and only if all prime ideals of height one are principal.
  \end{proposition}

  \begin{proposition}
    A subset $Y \subseteq \mathbb{A}^n$ is an affine variety with dimension $n-1$ if and only if it is the zero set of a non-constant irreducible polynomial in $A$.
  \end{proposition}
  \begin{proof}
    Let \(Y = Z(f)\) be the zero set of a non-constant irreducible polynomial \(f \in A\). \(A\) is a Noetherian integral domain, \(f\) is not a unit, and \((f)\) is a minimal prime ideal containing \(f\), so \(\operatorname{height}(f) = 1\). Therefore \(\dim Y = n - 1\).

    Conversely, let \(Y \subseteq \mathbb{A}^n\) be an affine variety with dimension \(n - 1\). Then \(\operatorname{height}I(Y) = 1\), so since \(A\) is a Noetherian unique factorisation domain, \(I(Y) = (f)\) for some irreducible \(f \in A\). Since \(\dim Y = n-1\), \(f\) is not a constant.
  \end{proof}
\end{document}
